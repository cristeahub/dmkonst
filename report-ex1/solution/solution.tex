\section{Solution}

% Solution: describes your solution of the task.
% Contains a detailed description of all the subtasks which have been solved and how they contribute to the solution for the given task.
% The use of diagrams, figures, tables and similar is welcome as a support to your description.

% TODO: Putt dette i introduksjonen? Eventuelt putt lista i et vedlegg
\subsection{Tools}

Development was done on computers in the computer lab running Ubuntu 12.04.
In addition, we used Microsoft Remote Desktop to be able to use Xilinx from other locations.
ghdl and gtkwave also proved useful, as it allowed us to work locally when only writing and testing vhdl modules. Figure \ref{fig:software} lists all the software used.

\begin{figure}[ht!]
    \begin{description}
        \item{\textbf{ISE Project Navigator 12.4 M.81d}}
            Main IDE for writing and synthesizing VHDL.
        \item{\textbf{ISim 12.4 M.81d}}
            Simulation environment for simulating VHDL.
        \item{\textbf{hostcomm 1.0.0}}
            Programming utility for the Spartan-6 LX16 Evaluation Kit.
        \item{\textbf{VIM}}
            For text editing.
        \item{\textbf{GHDL}}
            Allowed for writing and testing components on Mac OS X.
        \item{\textbf{gtkwave}}
            Visualization tool for inspecting test simulations run with ghdl.
        \item{\textbf{git \& GitHub}}
            Version control system with central code repository hosting.
    \label{fig:software}
    \end{description}
    \caption{Software used during the development process}
\end{figure}

\subsection{Architecture}

The suggested architecture for the project provided in Figure 4.1 [1, p.45] was abandoned in favor of a design based on the basic multi cycle architecture presented in lecture 4 [2].

As the host environment was tailored towards the single cycle MIPS architecture, notably including separate instruction and data memory (harvard architecture) as well as byte addressable memory, the architecture received slight modifications.

There's no dedicated component for executing memory reads and writes.
Instead, components responsible for computing memory addresses interface with memory directly, and incoming memory data buses are connected directly to instruction- and other registers.
As the external memory has one cycle response delay, some of the suggested datapath registers were removed.

Additional extensions to the architecture include support for the load upper immediate instruction, as well ass SLL and SRL functions.
The RTL sketch of the final design can be seen in figure \ref{fig:top_level_rtl}.

\begin{figure}[ht!]
    \begin{center}
    \includegraphics[width=1.2\textwidth]{assets/top_level_rtl.pdf}
    \caption{Top level RTL schematic. Outgoing lines are connected to external memory}
    \label{fig:top_level_rtl}
    \end{center}
\end{figure}

\subsubsection{MIPS subset}

The final processor design implements the following subset of the MIPS instruction set.

\begin{itemize}
    \item ALU instructions (ADD, SUB, SLT, AND, OR, SLL, SRL)
    \item Conditional Branch (BEZ)
    \item Jump Instruction (JMP)
    \item LOAD, STORE, Load immediate (LW, SW, LDI)
\end{itemize}

Note: the {\bf LDI} (Load Immediate) instruction is not a part of the MIPS instruction set\footnote{Figure C.1 \cite[p.66]{compendium}}.
The {\bf LUI} (Load Upper Immediate) is the closest match.
This instruction loads $ immediate\_value * 2 ^{16} $ into the register.
The support files handed out already use this instruction.

\subsubsection{Control unit}

The control unit is based on the state machine from lecture 4 \cite{lecture-4}.
It has been modified to add support for the extra load upper immediate instruction.
Output signals which were not needed in our architecture are also removed.
Specifically, the i\_or\_d and mem\_read signals were not needed, as our processor has separate instruction and data memories.

Because of the one cycle latency when reading from external memory, the ir\_write signal has been moved to the instruction\_decode state, when the correct instruction has been fetched.
This does not cause any latency issues, as the instruction register is designed to let the new instruction value pass through when ir\_write is high.
A state diagram for the control unit can be seen in figure \ref{fig:state_machine}.

The control unit takes two inputs, the current opcode (first 6 bits of the instruction) and a processor enable signal.
When the processor is disabled, the state machine should stay in the idle state.

The outputs from the state machine are dependent only on the current state.
This makes it a Moore machine.

\begin{figure}[ht!]
    \begin{center}
    \includegraphics[width=\textwidth]{assets/state_machine.pdf}
    \caption{State diagram for the control unit}
    \label{fig:state_machine}
    \end{center}
\end{figure}

\subsection{Implementation}

A modular approach was taken when implementing the system.
An interface was defined for each component, defining which inputs and outputs it should have,
and describing how it should perform.
Their functionality were verified with separate testbenches, which meant that they could be developed independently of each other.
This way, we were able to develop components in parallel.
Different components are detailed below.

\subsubsection{Instruction register}

The instruction register stores the instruction which is currently being executed.
To avoid unnecessary delay, the instruction\_out signal must be updated immediately when the instruction register receives a new instruction.
To achieve this, it includes a bypass function.
It is implemented using a combination of a flip-flop and a mux;
when the ir\_write signal from the control unit is high, the flip-flop updates its value,
but at the same time, the mux routes the instruction\_in signal directly to the out signal.

On the next cycle, when the ir\_write signal is low again, the flip-flop will contain the new value.
This creates the illusion that the instruction register updates instantly.

\subsubsection{Register file}

Stores data.

\subsubsection{ALU Control}

The behaviour of the ALU control unit is heavily based on the one described in lecture 4\cite{lecture-4}.
It has been extended to support the shift instructions SLL and SRL, therefore requiring the shamt part of the current instruction as input.
In case of an R-type instruction, the shamt value is forwarded straight to the ALU.
In the case of an LUI instruction, the alu should shift its input by 16 to the left.
Therefore "10000" is sent as shamt to the ALU, and ALU Control is set to SLL.
This reuse of generic components is rather nice!

Due to these extensions the hardware of the ALU Control has diverged from the simple design originally presented in the lecture slides.
To remedy this, the alu control out values were changed from the provided constants to an VHDL enum.
This allows the compiler to assign values that will result in more optimized hardware than the authors could design manually.

\begin{table}[ht!]
    \begin{tabular}{|l|l|l|l|l|l|l|}
    \hline
    Opcode & ALUOp & Operation           & Function & ALU Control & Shamt   \\ \hline
    lw     & 00    & load word           & XXXXXX   & ADD         & XXXXX \\ \hline
    sw     & 00    & store word          & XXXXXX   & ADD         & XXXXX \\ \hline
    beq    & 01    & branch equal        & XXXXXX   & SUBTRACT    & XXXXX \\ \hline
    r-type & 10    & add                 & 100000   & ADD         & XXXXX \\ \cline{3-6}
           &       & subtract            & 100010   & SUBTRACT    & XXXXX \\ \cline{3-6}
           &       & AND                 & 100100   & AND         & XXXXX \\ \cline{3-6}
           &       & OR                  & 100101   & OR          & XXXXX \\ \cline{3-6}
           &       & set-on-less-than    & 101010   & SLT         & XXXXX \\ \cline{3-6}
           &       & shift-left-logical  & 000000   & SLL         & Pass through \\ \cline{3-6}
           &       & shift-right-logical & 000010   & SRL         & Pass through \\ \hline
    lui    & 11    & lui                 & XXXXXX   & SLL         & 10000 \\ \hline
    \end{tabular}
    \caption{ALU control table}
    \label{tab:alu-control}
\end{table}

\subsection{Design optimization}

The initial register file implementation used 1024 flip-flops with the Mother of All Muxes connecting them together.
This can be a reasonable design choice given you're implementing your own IC, however large muxes can be expensive to implement on FPGAs.
With a comparable boost in clock frequency, the choice may still be worthwile.
This design ended up using 2530 lookup tables (16\% of available LUTs on the LX16), with Xilinx promptly advising that the usage of hardware resources such as block ram could be beneficial.
The final block ram register implementation ended up using 1022 fewer LUTs, along with increasing the maximum frequency of the processor from 60.562 MHz to 73.573 MHz.

As one of the authours thought sub 100 MHz clock frequencies were indicators of flaws in the design, significant time was spent on reducing critical paths through the system.
One of the earlier architectures did not have an instruction register, instead caching the current instruction memory address, keeping it constant for the entire execution of each instruction.
This increased the critical path for all instruction cycles, as Xilinx ISE inferred that the longest combinatorial sequence included signals propagating through external instruction memory.
Adding the instruction register ended up relieveing this issue, as well as opening for the possibility of disabling external memory when unneeded.
Sadly, no chip-select interface was exposed to the MIPS Processor.
