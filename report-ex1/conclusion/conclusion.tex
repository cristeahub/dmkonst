\section{Conclusion}

% Conclusion: a brief conclusion of the performed work.
% Round up the challenges and results.

All requirements given in the assignment were completed.
In addition, the processor is fairly efficient as several optimizations has been made after completing the initial design.

Our approach of first implementing support for the required instructions, and confirming that they fork on the FPGA, before making optimizations and adding extra instructions proved to be smart.
In the end, there was time for profiling of the processor to identify bottlenecks.
Removing the largest bottlenecks ended up increasing the clock frequency to 73.573MH.

The most difficult part of this assignment was getting familiar with writing VHDL code.
Writing small components by themselves and testing them with testbenches was easy enough,
but inspecting the generated RTL often showed that the written VHDL did not properly describe the intended components.
In addition, testing the final processor design was quite tedious, as there were many signals to keep track of to figure out where the processor diverged from desired behavior.

In conclusion, it was a challenging and educational assignment.
Things learned from it will be helpful both in the next assignment as well as in the course TDT4295.
