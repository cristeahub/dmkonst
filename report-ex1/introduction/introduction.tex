\section{Introduction}
\label{sec:intro}

``
Introduction: introduces the task of the assignment and the challenges it brings.
Also, it gives a brief introduction to how the task was approached and in which way the solution was reached.
``

This report presents a solution to Project \#1 of TDT4255 at NTNU.
The main requirement of the assignment is to make a multi cycle MIPS processor\cite[p.47]{compendium}. The design should be verified with VHDL testbenches, and finally tested by deploying it to the Xilinx Spartan-6 FPGA.

\subsection{Single- and multi cycle processors}

A single cycle processor is able to execute any of its instructions within a single clock period.
This results in simpler hardware, but restricts the processors clock period to the execution time of its slowest instruction.
Disadvantages include having idle datapath elements during some instructions, slow average execution time, and sometimes being required to duplicate execution units.

To work around these issues, the datapath can be partitioned into multiple register-separated elements.
Instructions now require multiple cycles to finish, but simple instructions are no longer bound by the execution time of more complex ones.
This design allows reuse of datapath elements during different execution stages, resulting in a smaller footprint.
Clock speed also increases significantly, as the critical path is reduced to the longest register-separated datapath segment.

\subsection{Requirements}
\label{subsec:req}
A suggested architecture was provided in Figure 4.1 \cite[p.45]{compendium}, but was abandoned in favor of the basic multi cycle architecture presented in lecture 4.\cite{lecture-4}

The final processor design implements the following subset of the MIPS instruction set.

\begin{itemize}
    \item ALU instructions (ADD, SUB, SLT, AND, OR, SLL, SRL)
    \item Conditional Branch (BEZ)
    \item Jump Instruction (JMP)
    \item LOAD, STORE, Load immediate (LW, SW, LDI)
\end{itemize}

Note: the {\bf LDI} (Load Immediate) instruction is not a part of the MIPS instruction set\footnote{Figure C.1 \cite[p.66]{compendium}}.
The {\bf LUI} (Load Upper Immediate) is the closest match.
This instruction loads the value into the upper 16 bits of the register.
The support files handed out already used this instruction.

The system is first tested and verified in software, before moving on to testing on hardware.
The testing of the system both in software and in hardware is detailed in section%\vref{sec:testing}.

\subsection{Challenges and procedure}

The assignment provides several tasks that needs solving, these are detailed in section.% TODO: \vref{sec:tasks}.
In addition to the requirements one goal was set for the assignment.
This goal is to make sure all code complies with the standard set forth by the industry
and Xilinx for the Spartan-6 FPGA.
This is to ensure that the generated code will be as optimized as possible,
which in turn will make the processor faster and better.

The assignment is solved by creating a top-level design for the processor which is then subdivided into different components which can be implemented separately.
A testbench is then written for each component, to make sure it works as intended.
These components are afterwards connected together and a new testbench is created to
make sure the system as a whole works according to the specification.
