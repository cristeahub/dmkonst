\section{Discussion}

% Discussion: Discuss the assignment and your achievements.
% You are free to critically assess your work – what could have been done better, which way you would choose to go if given the same task again etc.
% You can also provide feedback about the assignment itself.

% Here comes discussion. How did we do?

%TODO:  ville ha en liten introduksjon så vi ikke starter et kapittel med å starte et underkapittel
%       4. Discussion
%       4.1 Req
%       synes det er bedre om det er noe i mellom der
The finished product itself is of a very satifying nature.
All goals were met, which will be detailed in this section.
There are four main categories which will be talked about;
requirements, performance, energy efficiency and further work.

\subsection{Requirements}
All the requirements given in the exercise were met.
The processor implements a basic subset of the MIPS instruction set \cite[p.64]{compendium} as specified.
It has been tested on a top-level and each component is also tested in separation.
The processor is also successfully tested it on the FPGA using the hostcomm \cite{hostcomm} utility.

\subsection{Performance}
% TODO: det er en del ting som sies på nytt her, som ikke er så veldig bra, burde skrives om
Because of all the precausions done with performance in mind, the final product has a very high performance.
The maximum achieved clock speed is 78 MHz for the processor.
This is significantly better than the limit of 24 MHz set by the testing framework for the FPGA.

To make sure the performance is as good as possible, a lot of choices have been made during the development.
Most importantly, the fact that Xilinx does a lot of work in the optimizing department has been taken into concideration.
This means that the code is largely written to make sure Xilinx understands what the code is describing.
One example of this is the elimination of self written state values.
These are instead made as an enum which in terms make Xilinx change the values to whatever is best for the situation.

\subsection{Energy Efficiency}
There is not many ways to make the design very energy efficient.
The communication with the external memory only takes one cycle, so the processor always have somthing to do.
If, however, the communication took multiple cycles, the rest of the processor could sleep while waiting for the memory.

Another way to make the design more energy efficient is to make sure all parts of the processor is used all the time.
If one made sure every part of the processor is used at each clock cycle,
the total amount of clock cycles is reduced and less power is consumed.
This is known as a pipelining processor, which differes slightly from the design in this exercise.

\subsection{Further work}

There is still room for optimization, to further increase the clock speed.
Another interesting improvement would be to implement support for a wider range of the MIPS instruction set.
Having support for all the defined instructions, would for instance enable us to run all sorts of available MIPS programs on our processor.
