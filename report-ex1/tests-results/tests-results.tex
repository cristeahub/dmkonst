\section{Tests and Results}

% Results: presents the results: what has been successfully completed and what did not work.
% If any ways around it were found, provide them at this place. Every solution should be tested for its validity.
% This is the place where you will describe what kind of testing you have performed and what the outcome of your tests was.

\subsection{Performance}

Is it fast yet?

\subsection{Energy efficiency}

How much power does it consume when running on the FPGA?

\subsection{Testbenches}

Each component has been tested and verified with a separate testbench written in VHDL and simulated in ISim.

\begin{figure}[ht!]
    \begin{center}
    \includegraphics[width=\textwidth]{assets/isim/memory_read_cycle.png}
    \caption{Snapshot of the top level testbench when executing a load instruction}
    \label{fig:memory_read_cycle}
    \end{center}
\end{figure}

\begin{figure}[ht!]
    \begin{center}
    \includegraphics[width=\textwidth]{assets/isim/branch_completion_cycle.png}
    \caption{Snapshot of the top level testbench when running a branch instruction}
    \label{fig:branch_completion_cycle}
    \end{center}
\end{figure}

\begin{figure}[ht!]
    \begin{center}
    \includegraphics[width=\textwidth]{assets/isim/load_upper_cycle.png}
    \caption{Execution of a load immediate, where the value 393216 is stored in register 3}
    \label{fig:load_upper_cycle}
    \end{center}
\end{figure}

\begin{figure}[ht!]
    \begin{center}
    \includegraphics[width=\textwidth]{assets/isim/r_type_cycle.png}
    \caption{An r-type instruction with ALU op set to ADD}
    \label{fig:r_type_cycle}
    \end{center}
\end{figure}
