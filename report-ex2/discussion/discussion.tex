\chapter{Discussion}

\section{Requirements}
The major requirement, to implement a simple 5-stage pipelined processor was met.
Furthermore, the solution includes a data forwarding unit, a hazard detector, and a dynamic 2-bit branch predictor.
It has been successfully tested on a Xilinx Spartan-6 LX16 FPGA, using the provided hostcomm\cite{hostcomm} utility.

\section{Performance}
\todo{A branch predictor brings along quite the critical path, yadda yadda}

The final estimated clock frequency is about 60 MHz \todo{Har vi dette tallet liggende?}

When comparing to the multi-cycle processor we implemented in assignment 1, we see that the number of cycles required to execute the provided test program is reduced from 98 to 29.
And although the clock frequency has been somewhat reduced after introducing pipelining, it still represents a speedup of about 2.78.

\section{Energy Efficiency}

\section{The Work Process}
We decided to take a gradual approach to the assignment.
We first converted the processor from Assignment 1 into a processor with five pipeline stages, and modified the test program to have enough nops to remove all hazards.
With this implementation working, we could progress to add data forwarding, and observe that the test passed with significantly fewer nops required.
This approach let us test each new component's integration in the system before progressing to the new task, until finally the original test program ran to completion.

\section{Further Work}
Faster, better

The branch predictor can be quite easily extended.

Supporting a greater instruction set will allow us to run more useful programs.

\todo{dette sa vi sist gang også, gir det mening å ha med further work?}
