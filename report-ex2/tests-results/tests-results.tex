\chapter{Tests \& Results}

\section{Tests}

This section presents highlights from the testing suite of the pipelined MIPS processor with speculative execution. Branch prediction, data forwarding (including load-store forwarding), and pipeline flushing is demonstrated executing successfully.

\subsection{System integration test}

The provided system integration test from exercise 1 was successfully run on the pipelined architecture.
The test bench has been extended to be more thorough, checking that the data from store instructions following a branch taken don't end up in memory.

\subsection{Store after load operand forwarding}

The presented architecture has been extended to support operand forwarding from the write-back pipeline stage to the memory stage.
This allows for load instructions to be followed by store instructions without having to insert pipeline bubbles.

\begin{figure}[h]
  \begin{code}
    lw $3, 1($0)     # Loads some value from memory
    sw $3, 3($0)     # Should store the same value back to memory
  \end{code}
  \caption{Assembly to test store after load operand forwarding}
  \label{fig:test-store-after-load}
\end{figure}

Expected behavior:
\begin{itemize}
  \item
    The sw instruction should store the newly loaded value from the lw instruction, in effect acting like a memory move instruction.
  \item
    Bubbles should not be inserted into the pipeline.
\end{itemize}

\subsection{Branch prediction}

The test case in figure \ref{fig:test-branch-prediction} tests both the hazard detection and branch prediction capabilities of the processor.

\begin{figure}[h]
  \begin{code}
    lw $1, 0($0)
    lw $2, 1($0)      # Should stall after this instruction (data hazard).
    beq $2, $0, 3     # Should predict (Control hazard).
    sw $2, 2($0)
    sub $2, $2, $1    # Introduce data hazard for beq
    j 2               # Jump to loop condition
    beq $0, $0, -1    # Loop forever
  \end{code}
  \caption{Assembly to test branch prediction} \label{fig:test-branch-prediction}
\end{figure}

Expected behavior:
\begin{itemize}
  \item
    The pipeline stalling on instruction 2. Branch after load requires one stall cycle to actually be able to correct for wrongly taken branches.
  \item
    The processor predicting whether instruction 2 should branch or not.
    Operands are available neither the first time (load not having completed yet), or after the end-of-loop jump (the sub instruction not having completed yet).
    As this architecture doesn't support operand forwarding to the branch predictor, it will have to guess.
\end{itemize}


\subsection{Pipeline flushing}

The test case in figure \ref{fig:test-pipeline-flushing} tests that speculatively executed instructions are correctly flushed from the pipeline.

\begin{figure}[h]
  \begin{code}
    lw $2, 2($0)    # Loads 2 into $2
    lw $1, 1($0)    # Loads 1 into $1
    beq $1, $0, 2   # Branch speculatively not taken
    sw $2, 4($0)    # Speculatively executed, to be flushed
    j 6             # Speculatively executed, to be flushed
    sw $1, 4($0)    # Should actually be executed
  \end{code}
  \caption{Assembly to test pipeline flushing}
  \label{fig:test-pipeline-flushing}
\end{figure}

Expected behavior:
\begin{itemize}
  \item
    Instructions 3 and 4 should be speculatively executed, and flushed.
  \item
    The value from memory address 1, not 2 should end up at memory address 4.
\end{itemize}

