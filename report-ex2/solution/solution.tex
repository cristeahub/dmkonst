\chapter{Solution}

% Solution: describes your solution of the task.
% Contains a detailed description of all the subtasks which have been solved and how they contribute to the solution for the given task.
% The use of diagrams, figures, tables and similar is welcome as a support to your description.

\section{Architecture}

Data hazards, control hazards, structural hazards.

The design of the processor is based on the architecture provided in Figure 5.1 \cite[p. 50]{compendium}.
It has been extended to support jumps, data forwarding, hazard detection and branch prediction.

\section{Implementation}

\subsection{Data Forwarding}

\subsection {Hazard Detection}

\subsection{Branching}

This processor implements multiple branch-related improvements compared to a naïve pipelined datapath. These improvements allow for speculative execution of instructions, keeping the pipeline full at all times.

The improvements include:
\begin{enumerate}
  \item
    Moving branch decisions from the memory stage to the instruction decode stage, comparing register values in a dedicated equality unit.
    This allows for instant branches with no lost cycles when data hazards are absent.
  \item
    When operands are unavailable in the ID stage, one cannot decide whether to take the branch.
    By making an educated guess using a branch prediction unit (containing a two-bit predictor\todo{cite a two bit predictor explained somewhere else so we dont have to? Or should we explain it}) we can speculatively execute instructions, keeping the pipeline filled.
    If a branch is taken wrongly, only two pipeline stages have to be flushed, always netting an improvement over delayed branching.
\end{enumerate}

To allow for zero-downtime branching, the program counter allows for instant updates, implemented via an internal bypass straight through to the instruction memory.
This allows us to update the PC to the target branch address the same cycle we decide to branch, thereby having the branch-target instruction ready at the next clock cycle.

To correct for wrongly taken branches, all branch related signals (branch\_taken, control\_should\_branch, and branch\_pc\_not\_taken) are passed on through the pipeline.
In the MEM stage, when the branch operands actually have been compared, corrective action can be taken the wrong choice was made. If there was a choice to branch, and the branch taken signal and alu result zero flag differ, the branch was taken wrongly.
Pipeline stages ID and EX have to be flushed, as well as the pc reset to the address of the pc not taken (branch\_pc\_not\_taken).
Afterwards, execution resumes as normal, having lost only two cycles if the branch was predicted wrongly.

There are multiple ways to decide on which choice to predict.
\todo{This entire bit. finish it}
STATIC BRANCHING VERSUS DYNAMIC
This data is also fed back to the branch prediction unit, allowing the two bit predictor to learn from recent branch behaviour to give better predictions in the future.
tradeoff critical path thru branch predictor
